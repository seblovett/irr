%  PowerScaling.tex
%  Document created by seblovett on seblovett-NETBOOK
%  Date created: Wed 19 Feb 2014 11:28:42 GMT
%  <+Last Edited: Sat 22 Feb 2014 13:21:22 GMT by seblovett on seblovett-Ubuntu +>


\section{Power Scaling}
Lorem Ipsum\dots

